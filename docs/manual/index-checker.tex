\htmlhr
\chapterAndLabel{Index Checker for sequence bounds (arrays, strings, lists)}{index-checker}

The Index Checker warns about potentially out-of-bounds accesses to sequence
data structures, such as arrays, lists, and strings.

The Index Checker prevents \<IndexOutOfBoundsException>s that result from
an index expression that might be negative or might be equal to or larger
than the sequence's length.
It also prevents \<NegativeArraySizeException>s that result from a negative
array dimension in an array creation expression.
(A caveat: the Index Checker does not check for arithmetic overflow. If
an expression overflows, the Index Checker might fail to warn about a
possible exception.  This is unlikely to be a problem in practice unless
you have an array whose length is \<Integer.MAX\_VALUE>.)

% Here's a pathological example of overflow leading to unsoundness:
%
% public class IndexOverflow {
%     public static void main(String[] args) {
%         @Positive int x = 1073741825; // 2 ^ 30 + 1
%         @Positive int x2 = x + x; // 2 ^ 31 + 2 == - 2 ^ 31 + 2
%         int[] a = new int[0];
%         if (x2 < a.length) {
%             a[x2] = 42;
%         }
%     }
% }

The programmer can write annotations that indicate which expressions are
indices for which sequences.  The Index Checker prohibits any operation that
may violate these properties, and the Index Checker takes advantage of
these properties when verifying indexing operations.
%
Typically, a programmer writes few annotations, because the Index Checker
infers properties of indexes from
the code around them. For example, it will infer that \<x> is positive
within the \<then> block of an \code{if (x > 0)} statement.
The programmer does need to write field types and method pre-conditions or post-conditions. For instance,
if a formal parameter \<i> is used as an index for
\<myArray>, the programmer might need to declare it as
\refqualclasswithparams{checker/index/qual}{IndexFor}{"myArray"}~\<int~i>.

The Index Checker checks sequences that do not shrink; that is, their
size never decreases after creation.  There are two types of such sequences.
(1) Fixed-size sequences (with no \<add> or
\<remove> operation), such as strings and arrays.
You can add
support for other fixed-size data structures (see
Section~\ref{index-annotating-fixed-size}).
(2) Sequences that grow but do not shrink.  A program might use their
\<add> operation, but not their \<remove> operation.  An example is a
\<List> into which the program adds elements without removing any.

To run the Index Checker, run either of these commands:

\begin{alltt}
  javac -processor index \emph{MyJavaFile}.java
  javac -processor org.checkerframework.checker.index.IndexChecker \emph{MyJavaFile}.java
\end{alltt}

Recall that in Java, type annotations are written before the type;
in particular,
array annotations appear immediately before ``\<[]>''.
Here is how to declare a length-9 array of positive integers:

\begin{Verbatim}
  @Positive int @ArrayLen(9) []
\end{Verbatim}

Multi-dimensional arrays are similar.
Here is how to declare a length-2 array of length-4 arrays:

\begin{Verbatim}
  String @ArrayLen(2) [] @ArrayLen(4) []
\end{Verbatim}


\sectionAndLabel{Index Checker structure and annotations}{index-annotations}

Internally, the Index Checker computes information about integers that
might be indices:
\begin{itemize}
\item
  the lower bound on an integer, such as whether it is known to be positive
  (Section~\ref{index-lowerbound})
\item
  the upper bound on an integer, such as whether it is less than the length
  of a given sequence (Section~\ref{index-upperbound})
\item
  whether an integer came from calling the JDK's binary search routine on
  an array (Section~\ref{index-searchindex})
\item
  whether an integer came from calling a string search routine
  (Section~\ref{index-substringindex})
\end{itemize}

\noindent
and about sequence lengths:
\begin{itemize}
\item
  the minimum length of a sequence, such ``\<myArray> contains at least 3
  elements'' (Section~\ref{index-minlen})
\item
  whether two sequences have the same length (Section~\ref{index-samelen})
\end{itemize}

The Index Checker checks of all these properties at once, but
this manual discusses each type system in a different section.
There are some annotations that are shorthand for writing multiple
annotations, each from a different type system:

\begin{description}
\item[\refqualclasswithparams{checker/index/qual}{IndexFor}{String[] names}]
  The value is a valid index for the named sequences.  For example, the
  \sunjavadoc{java.base/java/lang/String.html\#charAt(int)}{String.charAt(int)}
  method is declared as

  \begin{Verbatim}
  class String {
    char charAt(@IndexFor("this") int index) { ... }
  }
  \end{Verbatim}

  More generally, a variable
  declared as \<@IndexFor("someArray") int i> has type
  \<@IndexFor("someArray") int> and its run-time value is guaranteed to be
  non-negative and less than the length of \<someArray>.  You could also
  express this as
  \<\refqualclass{checker/index/qual}{NonNegative}
  \refqualclasswithparams{checker/index/qual}{LTLengthOf}{"someArray"}
  int i>,
  but \<@IndexFor("someArray") int i> is more concise.

 \item[\refqualclasswithparams{checker/index/qual}{IndexOrHigh}{String[] names}]
   The value is non-negative and is less than or equal to the length of
   each named sequence.  This type combines
  \refqualclass{checker/index/qual}{NonNegative} and
  \refqualclass{checker/index/qual}{LTEqLengthOf}.

  For example, the
  \sunjavadoc{java.base/java/util/Arrays.html\#fill(java.lang.Object\%5B\%5D,int,int,java.lang.Object)}{Arrays.fill}
   method is declared as

  \begin{mysmall}
  \begin{Verbatim}
  class Arrays {
    void fill(Object[] a, @IndexFor("#1") int fromIndex, @IndexOrHigh("#1") int toIndex, Object val)
  }
  \end{Verbatim}
  \end{mysmall}

 \item[\refqualclasswithparams{checker/index/qual}{LengthOf}{String[] names}]
   The value is exactly equal to the length of the named
   sequences. In the implementation, this type aliases
   \refqualclass{checker/index/qual}{IndexOrHigh}, so writing it
   only adds documentation (although future versions of the Index Checker
   may use it to improve precision).

 \item[\refqualclasswithparams{checker/index/qual}{IndexOrLow}{String[] names}]
   The value is -1 or is a valid index for
   each named sequence.  This type combines
  \refqualclass{checker/index/qual}{GTENegativeOne} and
  \refqualclass{checker/index/qual}{LTLengthOf}.

%  Example commented out; IndexOrLow is not sound for indexOf, because "".indexOf("") returns 0
%
%  For example, the
%  \sunjavadoc{java.base/java/lang/String.html\#indexOf(java.lang.String)}{String.indexOf(String)}
%  method is declared as
%
%  \begin{Verbatim}
%  class String {
%    @IndexOrLow("this") int indexOf(String str) { ... }
%  }
%  \end{Verbatim}

 \item[\refqualclass{checker/index/qual}{PolyIndex}]
   indicates qualifier polymorphism.  This type combines
   \refqualclass{checker/index/qual}{PolyLowerBound} and
   \refqualclass{checker/index/qual}{PolyUpperBound}.
   For a description of qualifier polymorphism, see
   Section~\ref{method-qualifier-polymorphism}.

 \item[\refqualclass{checker/index/qual}{PolyLength}]
   is a special polymorphic qualifier that combines
   \refqualclass{checker/index/qual}{PolySameLen} and
   \refqualclass{common/value/qual}{PolyValue} from the
   Constant Value Checker (see \chapterpageref{constant-value-checker}).
   \refqualclass{checker/index/qual}{PolyLength} exists
   as a shorthand for these two annotations, since
   they often appear together.

\end{description}

\sectionAndLabel{Lower bounds}{index-lowerbound}

The Index Checker issues an error when
a sequence is indexed by an integer that might be negative.
The Lower Bound Checker uses a type system (Figure~\ref{fig-index-int-types}) with the following
qualifiers:

\begin{description}
\item[\refqualclass{checker/index/qual}{Positive}]
  The value is 1 or greater, so it is not too low to be used as an index.
  Note that this annotation is trusted by the Constant Value Checker,
  so if the Constant Value Checker is run on code containing this annotation,
  the Lower Bound Checker must be run on the same code in order to
  guarantee soundness.
\item[\refqualclass{checker/index/qual}{NonNegative}]
  The value is 0 or greater, so it is not too low to be used as an index.
\item[\refqualclass{checker/index/qual}{GTENegativeOne}]
  The value is -1 or greater.
  It may not be used as an index for a sequence, because it might be too low.
  (``\<GTE>'' stands for ``Greater Than or Equal to''.)
\item[\refqualclass{checker/index/qual}{PolyLowerBound}]
  indicates qualifier polymorphism.
  For a description of qualifier polymorphism, see
  Section~\ref{method-qualifier-polymorphism}.
\item[\refqualclass{checker/index/qual}{LowerBoundUnknown}]
  There is no information about the value.
  It may not be used as an index for a sequence, because it might be too low.
\item[\refqualclass{checker/index/qual}{LowerBoundBottom}]
    The value cannot take on any integral types. The bottom type, which
    should not need to be written by the programmer.
\end{description}

\begin{figure}
\begin{center}
  \hfill
  \includeimagenocentering{lowerbound}{5cm}
  ~~~~\hfill~~~~
  \includeimagenocentering{upperbound}{7cm}
  \hfill
\end{center}
  \caption{The two type hierarchies for integer types used by the Index
    Checker.  On the left is a type system for lower bounds.  On the right
    is a type system for upper bounds.  Qualifiers written in gray should
    never be written in source code; they are used internally by the type
    system.
    % Using "\\" works for some but not all installations of LaTeX.
    \newline
    In the Upper Bound type system, subtyping rules depend on both the
    array name (\<"myArray">, in the figure) and on the offset (which is 0,
    the default, in the figure).  Another qualifier is
    \refqualclass{checker/index/qual}{UpperBoundLiteral}, whose subtyping
    relationships depend on its argument and on offsets for other qualifiers.
 }
  \label{fig-index-int-types}
\end{figure}


\sectionAndLabel{Upper bounds}{index-upperbound}

The Index Checker issues an error when a sequence index might be
too high. To do this, it maintains information about which expressions are
safe indices for which sequences.
The length of a sequence is \code{arr.length} for arrays and
\code{str.length()} for strings.
It uses a type system (Figure~\ref{fig-index-int-types}) with the following
qualifiers:

It issues an error when a sequence \code{arr}
is indexed by an integer that is not of type \code{@LTLengthOf("arr")}
or \code{@LTOMLengthOf("arr")}.

\begin{description}

\item[\refqualclasswithparams{checker/index/qual}{LTLengthOf}{String[] names, String[] offset}]
  An expression with this type
  has value less than the length of each sequence listed in \<names>.
  The expression may be used as an index into any of those sequences,
  if it is non-negative.
  For example, an expression of type \code{@LTLengthOf("a") int} might be
  used as an index to \<a>.
  The type \code{@LTLengthOf(\{"a", "b"\})} is a subtype of both
  \code{@LTLengthOf("a")} and \code{@LTLengthOf("b")}.
  (``\<LT>'' stands for ``Less Than''.)

  \<@LTLengthOf> takes an optional \<offset> element, meaning that the
  annotated expression plus the offset is less than the length of the given
  sequence.  For example, suppose expression \<e> has type \<@LTLengthOf(value
  = \{"a", "b"\}, offset = \{"-1", "x"\})>. Then \<e - 1> is less than
  \<a.length>, and \<e + x> is less than \<b.length>.  This helps to make
  the checker more precise.  Programmers rarely need to write the \<offset>
  element.

\item[\refqualclasswithparams{checker/index/qual}{LTEqLengthOf}{String[] names}]
  An expression with this type
  has value less than or equal to the length of each sequence listed in \<names>.
  It may not be used as an index for these sequences, because it might be too high.
  \code{@LTEqLengthOf(\{"a", "b"\})} is a subtype of both
  \code{@LTEqLengthOf("a")} and \code{@LTEqLengthOf("b")}.
  (``\<LTEq>'' stands for ``Less Than or Equal to''.)

  \<@LTEqLengthOf(\{"a"\})> = \<@LTLengthOf(value=\{"a"\}, offset=-1)>, and \\
  \<@LTEqLengthOf(value=\{"a"\}, offset=x)> = \<@LTLengthOf(value=\{"a"\},
  offset=x-1)> for any x.

\item[\refqualclasswithparams{checker/index/qual}{LTOMLengthOf}{String[] names}]
  An expression with this type
  has value at least 2 less than the length of each sequence listed in \<names>.
  It may always used as an index for a sequence listed in \<names>, if it is
  non-negative.

  This type exists to allow the checker to infer the safety of loops of
  the form:
\begin{Verbatim}
  for (int i = 0; i < array.length - 1; ++i) {
    arr[i] = arr[i+1];
  }
\end{Verbatim}
  This annotation should rarely (if ever) be written by the programmer; usually
  \refqualclasswithparams{checker/index/qual}{LTLengthOf}{String[] names}
  should be written instead.
  \code{@LTOMLengthOf(\{"a", "b"\})} is a subtype of both
  \code{@LTOMLengthOf("a")} and \code{@LTOMLengthOf("b")}.
  (``\<LTOM>'' stands for ``Less Than One Minus'', because another way of
  saying ``at least 2 less than \<a.length>'' is ``less than \<a.length-1>''.)

  \<@LTOMLengthOf(\{"a"\})> = \<@LTLengthOf(value=\{"a"\}, offset=1)>, and \\
  \<@LTOMLengthOf(value=\{"a"\}, offset=x)> = \<@LTLengthOf(value=\{"a"\},
  offset=x+1)> for any x.

\item[\refqualclass{checker/index/qual}{UpperBoundLiteral}]
  represents a constant value, typically a literal written in source code.
  Its subtyping relationship is:
  \<@UpperBoundLiteral(lit)> <: \<LTLengthOf(value="myArray", offset=off)>
  if \<lit>+\<offset> $\le$ -1.

\item[\refqualclass{checker/index/qual}{PolyUpperBound}]
  indicates qualifier polymorphism.
  For a description of qualifier polymorphism, see
  Section~\ref{method-qualifier-polymorphism}.

\item[\refqualclass{checker/index/qual}{UpperBoundUnknown}]
  There is no information about the upper bound on the value of an expression with this type.
  It may not be used as an index for a sequence, because it might be too high.
  This type is the top type, and should never need to be written by the
  programmer.

\item[\refqualclass{checker/index/qual}{UpperBoundBottom}]
  This is the bottom type for the upper bound type system. It should
  never need to be written by the programmer.

\end{description}

The following method annotations can be used to establish a method postcondition
that ensures that a certain expression is a valid index for a sequence:

\begin{description}
\item[\refqualclasswithparams{checker/index/qual}{EnsuresLTLengthOf}{String[] value, String[] targetValue, String[] offset}]
  When the method with this annotation returns, the expression (or all the expressions) given in the \code{value} element
  is less than the length of the given sequences with the given offsets. More precisely, the expression
  has the \code{@LTLengthOf} qualifier with the \code{value} and \code{offset} arguments
  taken from the \code{targetValue} and \code{offset} elements of this annotation.
\item[\refqualclasswithparams{checker/index/qual}{EnsuresLTLengthOfIf}{String[] expression, boolean result, String[] targetValue, String[] offset}]
  If the method with this annotation returns the given boolean value,
  then the given expression (or all the given expressions)
  is less than the length of the given sequences with the given offsets.
\end{description}

There is one declaration annotation that indicates the relationship between
two sequences:

\begin{description}
\item[\refqualclasswithparams{checker/index/qual}{HasSubsequence}{String[]
    value, String[] from, String[] to}]
  indicates that a subsequence (from \code{from} to \code{to}) of the
  annotated sequence is equal to some other sequence, named by
  \code{value}).

For example, to indicate that \<shorter> is a subsequence of \<longer>:

\begin{Verbatim}
  int start;
  int end;
  int[] shorter;
  @HasSubsequence(value="shorter", from="this.start", to="this.end")
  int[] longer;
\end{Verbatim}

Thus, a valid index into \<shorter> is also a valid index (between
\code{start} and \code{end-1} inclusive) into \<longer>.  More generally,
if \code{x} is \code{@IndexFor("shorter")} in the example above, then
\code{start + x} is \code{@IndexFor("longer")}. If \code{y} is
\code{@IndexFor("longer")} and \code{@LessThan("end")}, then \code{y -
  start} is \code{@IndexFor("shorter")}. Finally, \code{end - start} is
\code{@IndexOrHigh("shorter")}.

This annotation is in part checked and in part trusted.  When an array is
assigned to \code{longer}, three facts are checked: that \code{start} is
non-negative, that \code{start} is less than or equal to \code{end}, and
that \code{end} is less than or equal to the length of \code{longer}.  This
ensures that the indices are valid. The programmer must manually verify
that the value of \code{shorter} equals the subsequence that they describe.
\end{description}


\sectionAndLabel{Sequence minimum lengths}{index-minlen}

The Index Checker estimates, for each sequence expression, how long its value
might be at run time by computing a minimum length that
the sequence is guaranteed to have.  This enables the Index Checker to
verify indices that are compile-time constants.  For example, this code:

\begin{Verbatim}
  String getThirdElement(String[] arr) {
    return arr[2];
  }
\end{Verbatim}

\noindent
is legal if \<arr> has at least three elements, which can be indicated
in this way:

\begin{Verbatim}
  String getThirdElement(String @MinLen(3) [] arr) {
    return arr[2];
  }
\end{Verbatim}

When the index is not a compile-time constant, as in \<arr[i]>, then the
Index Checker depends not on a \<@MinLen> annotation but on \<i> being
annotated as
\refqualclasswithparams{checker/index/qual}{LTLengthOf}{"arr"}.

The MinLen type qualifier is implemented in practice by the Constant Value Checker,
using \<@ArrayLenRange> annotations (see \chapterpageref{constant-value-checker}).
This means that errors related to the minimum lengths of arrays must be suppressed using
the "value" argument to \<@SuppressWarnings>.
\refqualclass{common/value/qual}{ArrayLenRange} and \refqualclass{common/value/qual}{ArrayLen}
annotations can also be used to establish the minimum length of a sequence, if a
more precise estimate of length is known. For example,
if \<arr> is known to have exactly three elements:

\begin{Verbatim}
  String getThirdElement(String @ArrayLen(3) [] arr) {
    return arr[2];
  }
\end{Verbatim}

The following type qualifiers (from \chapterpageref{constant-value-checker})
can establish the minimum length of a sequence:

\begin{description}
\item[\refqualclasswithparams{common/value/qual}{MinLen}{int value}]
  The value of an expression of this type is a sequence with at least
  \code{value} elements.  The default annotation is
  \code{@MinLen(0)}, and it may be applied to non-sequences.
  \code{@MinLen($x$)} is a subtype of \code{@MinLen($x-1$)}.
  An \code{@MinLen} annotation is treated internally as an
  \refqualclass{common/value/qual}{ArrayLenRange} with only its
  \code{from} field filled.
\item[\refqualclasswithparams{common/value/qual}{ArrayLen}{int[] value}]
  The value of an expression of this type is a sequence whose
  length is exactly one of the integers listed in its argument.
  The argument can contain at most ten integers; larger collections of
  integers are converted to \refqualclass{common/value/qual}{ArrayLenRange}
  annotations. The minimum length of a sequence with this annotation
  is the smallest element of the argument.
\item[\refqualclasswithparams{common/value/qual}{ArrayLenRange}{int from, int to}]
  The value of an expression of this type is a sequence whose
  length is bounded by its arguments, inclusive.
  The minimum length of a sequence with this annotation is its \<from> argument.
\end{description}

\begin{figure}
\begin{center}
  \hfill
  \includeimage{samelen}{5cm}
  \hfill
\end{center}
  \caption{The type hierarchy for arrays of equal length ("a" and "b" are
    assumed to be in-scope sequences).  Qualifiers
    written in gray should never be written in source code; they are used
    internally by the type system.}
  \label{fig-index-array-types}
\end{figure}

The following method annotation can be used to establish a method postcondition
that ensures that a certain sequence has a minimum length:

\begin{description}
\item[\refqualclasswithparams{common/value/qual}{EnsuresMinLenIf}{String[] expression, boolean result, int targetValue}]
  If the method with this annotation returns the given boolean value,
  then the given expression (or all the given expressions) is a sequence
  with at least \code{targetValue} elements.
\end{description}

\sectionAndLabel{Sequences of the same length}{index-samelen}

The Index Checker determines whether two or more sequences have the same length.
This enables it to verify that all the indexing operations are safe in code
like the following:

\begin{Verbatim}
  boolean lessThan(double[] arr1, double @SameLen("#1") [] arr2) {
    for (int i = 0; i < arr1.length; i++) {
      if (arr1[i] < arr2[i]) {
        return true;
      } else if (arr1[i] > arr2[i]) {
        return false;
      }
    }
    return false;
  }
\end{Verbatim}

When needed, you can specify which sequences have the same length using the following type qualifiers (Figure~\ref{fig-index-array-types}):

\begin{description}
\item[\refqualclasswithparams{checker/index/qual}{SameLen}{String[] names}]
  An expression with this type represents a sequence that has the
  same length as the other sequences named in \<names>. In general,
  \code{@SameLen} types that have non-intersecting sets of names
  are \textit{not} subtypes of each other. However, if at least one
  sequence is named by both types, the types are actually the same,
  because all the named sequences must have the same length.
\item[\refqualclass{checker/index/qual}{PolySameLen}]
  indicates qualifier polymorphism.
  For a description of qualifier polymorphism, see
  Section~\ref{method-qualifier-polymorphism}.
\item[\refqualclass{checker/index/qual}{SameLenUnknown}]
  No information is known about which other sequences have the same length
  as this one.
  This is the top type, and programmers should never need to write it.
\item[\refqualclass{checker/index/qual}{SameLenBottom}]
  This is the bottom type, and programmers should rarely need to write it.
  \code{null} has this type.
\end{description}


\sectionAndLabel{Binary search indices}{index-searchindex}

The JDK's
\sunjavadoc{java.base/java/util/Arrays.html\#binarySearch(java.lang.Object\%5B\%5D,java.lang.Object)}{Arrays.binarySearch}
method returns either where the value was found, or a negative value
indicating where the value could be inserted.  The Search Index Checker
represents this concept.

\begin{figure}
\begin{center}
  \hfill
  \includeimage{searchindex}{7cm}
  \hfill
\end{center}
  \caption{The type hierarchy for the Index Checker's internal type system
  that captures information about the results of calls to
  \sunjavadoc{java.base/java/util/Arrays.html\#binarySearch(java.lang.Object\%5B\%5D,java.lang.Object)}{Arrays.binarySearch}.}
  \label{fig-index-searchindex}
\end{figure}

The Search Index Checker's type hierarchy (Figure~\ref{fig-index-searchindex}) has four type qualifiers:
\begin{description}
\item[\refqualclasswithparams{checker/index/qual}{SearchIndexFor}{String[] names}]
  An expression with this type represents an integer that could have been
  produced by calling
  \sunjavadoc{java.base/java/util/Arrays.html\#binarySearch(java.lang.Object\%5B\%5D,java.lang.Object)}{Arrays.binarySearch}:
  for each array \<a> specified in the annotation, the annotated integer is
  between \<-a.length-1> and \<a.length-1>, inclusive
\item[\refqualclasswithparams{checker/index/qual}{NegativeIndexFor}{String[] names}]
  An expression with this type represents a ``negative index'' that is
  between \<a.length-1> and \<-1>, inclusive; that is, a value that is both
  a \<@SearchIndex> and is negative.  Applying the bitwise complement
  operator (\verb|~|) to an expression of this type produces an expression
  of type \refqualclass{checker/index/qual}{IndexOrHigh}.
\item[\refqualclass{checker/index/qual}{SearchIndexBottom}]
  This is the bottom type, and programmers should rarely need to write it.
\item[\refqualclass{checker/index/qual}{SearchIndexUnknown}]
  No information is known about whether this integer is a search index.
  This is the top type, and programmers should rarely need to write it.
\end{description}


\sectionAndLabel{Substring indices}{index-substringindex}

The methods
\sunjavadoc{java.base/java/lang/String.html\#indexOf(java.lang.String)}{String.indexOf}
and
\sunjavadoc{java.base/java/lang/String.html\#lastIndexOf(java.lang.String)}{String.lastIndexOf}
return an index of a given substring within a given string, or -1 if no
such substring exists.  The index \<i> returned from
\<receiver.indexOf(substring)> satisfies the following property, which is
stated here in three equivalent ways:
\begin{Verbatim}
 i == -1 || ( i >= 0       && i <= receiver.length() - substring.length()                  )
 i == -1 || ( @NonNegative && @LTLengthOf(value="receiver", offset="substring.length()-1") )
 @SubstringIndexFor(value="receiver", offset="substring.length()-1")
\end{Verbatim}

% This new annotation is similar to \<@LTLengthOf\allowbreak(value =
% "receiver", offset = "substring.length()-1")>, but explicitly allows the
% index to be -1 even if the upper bound would not allow it because of the
% offset.  The Upper Bound Checker can infer the corresponding
% \<@LTLengthOf> annotation for expressions that have a
% \<@SubstringIndexFor> annotation and at the same time are known to be
% non-negative (according to the Lower Bound Checker).

The return type of methods \sunjavadoc{java.base/java/lang/String.html\#indexOf(java.lang.String)}{String.indexOf}
and \sunjavadoc{java.base/java/lang/String.html\#lastIndexOf(java.lang.String)}{String.lastIndexOf} has the annotation
\refqualclasswithparams{checker/index/qual}{SubstringIndexFor}{value="this", offset="\#1.length()-1")}.
This allows writing code such as the following with no warnings from the
Index Checker:

\begin{Verbatim}
  public static String removeSubstring(String original, String removed) {
    int i = original.indexOf(removed);
    if (i != -1) {
      return original.substring(0, i) + original.substring(i + removed.length());
    }
    return original;
  }
\end{Verbatim}

% The code removes the first occurrence of \<removed> from
% \<original>. After checking that \code{i != -1}, the value of \<i> must
% be a valid index for \<original>. Because this index is the start of an
% occurrence of \<removed>, \code{i + removed.length()} is the index of the
% end of the occurrence.  Without the \<@SubstringIndexFor> annotation, the
% Upper Bound Checker would not be able to verify that \code{i +
% removed.length()} is a valid argument to \<substring>, which requires
% both arguments to be \<@IndexOrHigh("original")>.

\begin{figure}
\begin{center}
  \hfill
  \includeimage{substringindex}{3.5cm}
  \hfill
\end{center}
  \caption{The type hierarchy for the Substring Index Checker, which
    captures information about the results of calls to
    \sunjavadoc{java.base/java/lang/String.html\#indexOf(java.lang.String)}{String.indexOf}
    and
    \sunjavadoc{java.base/java/lang/String.html\#lastIndexOf(java.lang.String)}{String.lastIndexOf}.}
  \label{fig-index-substringindex}
\end{figure}

The \<@SubstringIndexFor> annotation is implemented in a Substring Index
Checker that runs together with the Index Checker and has its own type
hierarchy (Figure~\ref{fig-index-substringindex}) with three type
qualifiers:
\begin{description}
\item[\refqualclasswithparams{checker/index/qual}{SubstringIndexFor}{String[] value, String[] offset}]
  An expression with this type represents an integer that could have been
  produced by calling
  \sunjavadoc{java.base/java/lang/String.html\#indexOf(java.lang.String)}{String.indexOf}:
  the annotated integer is either -1, or it is non-negative and is less
  than or equal to \<receiver.length - offset> (where the sequence
  \<receiver> and the offset \<offset> are corresponding elements of the
  annotation's arguments).
\item[\refqualclass{checker/index/qual}{SubstringIndexBottom}]
  This is the bottom type, and programmers should rarely need to write it.
\item[\refqualclass{checker/index/qual}{SubstringIndexUnknown}]
  No information is known about whether this integer is a substring index.
  This is the top type, and programmers should rarely need to write it.
\end{description}


\subsectionAndLabel{The need for the \<@SubstringIndexFor> annotation}{index-substringindex-justification}

No other annotation supported by the Index Checker precisely represents the
possible return values of methods
\sunjavadoc{java.base/java/lang/String.html\#indexOf(java.lang.String)}{String.indexOf}
and
\sunjavadoc{java.base/java/lang/String.html\#lastIndexOf(java.lang.String)}{String.lastIndexOf}.
The reason is the methods' special cases for empty strings and for failed matches.

Consider the result \<i> of \<receiver.indexOf(substring)>:

\begin{itemize}
\item
  \<i> is \<@GTENegativeOne>, because \code{i >= -1}.
\item
  \<i> is \<@LTEqLengthOf("receiver")>, because \code{i <= receiver.length()}.
\item
  \<i> is not \<@IndexOrLow("receiver")>, because for
  \code{receiver = "", substring = "", i = 0}, the property
  \code{i >= -1 \&\& i < receiver.length()} does not hold.
\item
  \<i> is not \<@IndexOrHigh("receiver")>, because for
  \code{receiver = "", substring = "b", i = -1}, the property
  \code{i >= 0 \&\& i <= receiver.length()} does not hold.
\item
  \<i> is not
  \<@LTLengthOf(value = "receiver", offset = "substring.length()-1")>,
  because for \code{receiver = "", substring = "abc", i = -1}, the property
  \code{i + substring.length() - 1 < receiver.length()} does not hold.
\end{itemize}

\noindent
The last annotation in the list above,
\<@LTLengthOf(value = "receiver", offset = "substring.length()-1")>,
is the correct and precise upper bound for all values of \<i> except -1.
The offset expresses the fact that we can add \<substring.length()> to this
index and still get a valid index for \<receiver>.  That is useful for
type-checking code that adds the length of the substring to the found
index, in order to obtain the rest of the string.  However, the upper bound
applies only after the index is explicitly checked not to be -1:

\begin{Verbatim}
  int i = receiver.indexOf(substring);
  // i is @GTENegativeOne and @LTEqLengthOf("receiver")
  // i is not @LTLengthOf(value = "receiver", offset = "substring.length()-1")
  if (i != -1) {
    // i is @NonNegative and @LTLengthOf(value = "receiver", offset = "substring.length()-1")
    int j = i + substring.length();
    // j is @IndexOrHigh("receiver")
    return receiver.substring(j); // this call is safe
  }
\end{Verbatim}

The property of the result of \<indexOf> cannot be expressed by any
combination of lower-bound (Section~\ref{index-lowerbound}) and upper-bound
(Section~\ref{index-upperbound}) annotations, because the upper-bound
annotations apply independently of the lower-bound annotations, but in this
case, the upper bound \code{i <= receiver.length() - substring.length()}
holds only if \code{i >= 0}.  Therefore, to express this property and make
the example type-check without false positives, a new annotation such as
\<@SubstringIndexFor\allowbreak(value = "receiver", offset = "substring.length()-1")>
is necessary.

\sectionAndLabel{Inequalities}{index-inequalities}

The Index Checker estimates which expression's values are less than other expressions' values.

\begin{description}

\item[\refqualclasswithparams{checker/index/qual}{LessThan}{String[] values}]
  An expression with this type has a value that is less than the value of each
  expression listed in \<values>. The expressions in values must be composed of
  final or effectively final variables and constants.

\item[\refqualclass{checker/index/qual}{LessThanUnknown}]
  There is no information about the value of an expression this type relative to other expressions.
  This is the top type, and should not be written by the programmer.

 \item[\refqualclass{checker/index/qual}{LessThanBottom}]
   This is the bottom type for the less than type system. It should
   never need to be written by the programmer.

\end{description}

\sectionAndLabel{Annotating your own fixed-size datatypes}{index-annotating-fixed-size}

The Index Checker has built-in support for Strings and arrays.
You can add support for additional fixed-size data structures by writing
annotations.
This allows the Index Checker to typecheck the data structure's
implementation and to typecheck uses of the class.

This section gives an example:  a fixed-length collection.

%% The code that follows is copied from checker/tests/index/ArrayWrapper.java.
%% If this code is updated, please update that file, too.

\begin{Verbatim}
/** ArrayWrapper is a fixed-size generic collection. */
public class ArrayWrapper<T> {
    private final Object @SameLen("this") [] delegate;

    @SuppressWarnings("index") // constructor creates object of size @SameLen(this) by definition
    ArrayWrapper(@NonNegative int size) {
        delegate = new Object[size];
    }

    public @LengthOf("this") int size() {
        return delegate.length;
    }

    public void set(@IndexFor("this") int index, T obj) {
        delegate[index] = obj;
    }

    @SuppressWarnings("unchecked") // required for normal Java compilation due to unchecked cast
    public T get(@IndexFor("this") int index) {
        return (T) delegate[index];
    }
}
\end{Verbatim}

The Index Checker treats methods annotated with \code{@LengthOf("this")}  as
the length of a sequence like \code{arr.length} for arrays and
\code{str.length()} for strings.

With these annotations, client code like the following typechecks with no
warnings:
\begin{Verbatim}
    public static void clearIndex1(ArrayWrapper<? extends Object> a, @IndexFor("#1") int i) {
        a.set(i, null);
    }

    public static void clearIndex2(ArrayWrapper<? extends Object> a, int i) {
        if (0 <= i && i < a.size()) {
            a.set(i, null);
        }
    }
\end{Verbatim}


\sectionAndLabel{Support for mutable-length data structures}{index-mutable-length}

The Index Checker, initially designed for \emph{fixed-size} sequences, has been
expanded to handle mutable-length data structures like collections like {\tt List}.
This includes addressing mutation and aliasing challenges, allowing safe mutations
like collection growth while restricting or checking mutations that shrink the
collection's length to maintain index safety. Challenges of mutation and aliasing:
\begin{itemize}
\item
  The Index Checker assumes no aliasing or length changes, which can lead to incorrect
  treatment of previously-validated indexes even after the collection has \textbf{shrunk}.
  This is because a collection's length can change, and an element removed from a list
  can invalidate an index. This issue is exacerbated by aliasing, where mutations via
  one reference can affect others. Therefore, it is crucial to address this issue.
\item
  The extended Index Checker must consider certain operations as potential invalidations
  of index information to maintain soundness. \textbf{Growing} a collection doesn't
  invalidate existing indices, while \textbf{shrinking} a collection can make previously
  valid indices too high. The type system distinguishes between possible length decreases
  and those not, and invalidates index facts upon operations that could reduce a collection's
  length or disrupt earlier assumptions.
\end{itemize}

\subsectionAndLabel{Permitting growth and restricting shrinking}{index-mutable-approach}

The extended Index Checker allows mutations that increase length with minimal overhead while
restricting operations that might decrease it. This allows programmers to append to a
collection without breaking index safety, but removals must be tracked and often require
revalidation of indices. A set of type qualifiers on collection references encodes permitted
mutations, allowing programmers to communicate to the type checker which references will
be used to modify lengths and how:
\begin{itemize}
\item
  A reference that is used only for reading or growing the collection
  (never to remove elements) can be treated almost like a fixed-size collection,
  since it will never witness a length-decreasing mutation.
\item
  A reference through which elements may be removed (shrinking the collection)
  is considered more volatile – the checker must enforce additional checks and
  may invalidate some facts after mutations.
\item
  In special cases, a programmer can declare that a reference’s mutations are not
  being checked for index safety (opting out for difficult scenarios or incremental
  adoption).
\end{itemize}

The type system prevents unsafe scenarios like using an index after an element removal
by allowing references to grow or not mutate and requiring explicit annotation for those
that can shrink, and defines a qualifier hierarchy to implement these ideas.

\subsectionAndLabel{Qualifier hierarchy for mutable-length collections}{index-mutable-qualifiers}

The type system contains the following type qualifiers
(Figure~\ref{index-mutability-lattice} shows their hierarchy).

\begin{description}
\item[\refqualclass{checker/index/qual}{UnShrinkableRef}]
  Ensures that the collection's size does not decrease as long as it is used exclusively.
  This is the \textbf{default qualifier} for mutable-length structures, and if no annotation is
  written, a reference is assumed to be @UnShrinkableRef. It restricts the \emph{current reference’s}
  usage, preventing removal (such as \code{remove()} or \code{clear()}), but does \emph{not} guarantee
  that the collection's size cannot change via other references. The Index Checker allows index-safe
  operations under the assumption that this reference doesn't shrink the sequence, but remains cautious
  about possible external mutations.

\item[\refqualclass{checker/index/qual}{GrowOnly}]
  Used to \emph{grow} the collection, preventing shrinking methods and ensuring no other alias
  can shrink it. It prevents coexistence with references that permit shrinking, ensuring valid
  indexes remain valid unless reassigned. This type system ensures safe appending without
  invalidating existing relationships and prevents the checker from unnecessarily
  invalidating index information after additions. Any attempt to call a removal method on a
  @GrowOnly reference is a type error.

\item[\refqualclass{checker/index/qual}{Shrinkable}]
  The Index Checker allows calls to element-removing methods through a reference that is @Shrinkable.
  However, this can reduce the length of the collection, so the checker takes extra precautions.
  It typically requires code to re-establish index bounds after a mutation and
  \emph{invalidates any previously inferred index facts} for the collection when a shrink operation occurs.
  A @Shrinkable reference also signifies that other aliases might shrink the collection.
  It is used only when needed and should be annotated as @Shrinkable if you intend to call methods that
  remove elements through it.

\item[\refqualclass{checker/index/qual}{UncheckedShrinkable}]
  @UncheckedShrinkable is a type checker that allows shrinking but \emph{doesn't guarantee} index safety.
  It assumes all indices are valid, even across mutations, and is used for cases where aliasing or
  mutation patterns are too complex for the current type system or during incremental adoption.
  This qualifier is rarely used in application code, as full checking is impractical. A reference
  annotated @UncheckedShrinkable is treated as potentially shrinkable but doesn't issue warnings for
  index operations (making it the programmer’s responsibility to ensure safety). It can be cast to
  @Shrinkable or vice versa, but it means relinquishing or reinstating compile-time guarantees.
  Using @UncheckedShrinkable trades soundness for expedience.

\item[\refqualclass{checker/index/qual}{BottomGrowShrink}]
  Used internally by the type system to represent the most restrictive reference — one that
  permits neither growth nor shrinking of a collection. This qualifier is not intended
  to be written in source code and will typically only appear in diagnostics or in
  subtyping relationships as part of type inference. As with other bottom types such as
  \refqualclass{checker/index/qual}{LowerBoundBottom} or
  \refqualclass{checker/index/qual}{UpperBoundBottom}, its main role is to serve as
  the subtype of all other mutability qualifiers. The type system ensures that no
  reference in a user program will be assigned this type, and it should never be used
  explicitly in annotations.
\end{description}

The top type is @UnShrinkableRef, which promises not to shrink the
collection. Both @Shrinkable and @GrowOnly are sub-types of @UnShrinkableRef, adding stronger
conditions for mutations. @GrowOnly is incompatible with @Shrinkable, representing distinct
mutation permissions. The @UncheckedShrinkable qualifier is a special case of @Shrinkable that
bypasses checking. An implicit "bottom" qualifier represents a reference that can neither grow nor
shrink. Assignments and method calls must respect this hierarchy. For example, a @Shrinkable collection
can be used in any context that expects an @UnShrinkableRef, but an @UnShrinkableRef cannot be passed
where a @Shrinkable is required. The type system prevents @GrowOnly references from being aliased by or
assigned to @Shrinkable references, as this would violate the guarantee that no shrink will occur.

\begin{figure}
\begin{center}
  \hfill
  \includeimage{index-mutability-lattice}{5cm}
  \hfill
\end{center}
  \caption{Qualifier hierarchy for Index Checker with mutable-length support.
    Arrows denote subtype relationships (lower means stricter/safer).}
  \label{index-mutability-lattice}
\end{figure}

The Index Checker treats unannotated collection references as \refqualclass{checker/index/qual}{UnShrinkableRef},
assuming that code won't remove elements via those references. This default is chosen because most collection
operations are additions or reads, and fewer places actually remove elements. If a mutating method shrinks
the collection without the reference being annotated, the checker will issue an error. To minimize annotation
burden, only add annotations in parts of code that perform removals or require special behavior.

\subsectionAndLabel{Aliasing and effect annotations}{index-mutable-aliasing}

A critical aspect of supporting mutable-length structures is handling \textbf{aliasing}:
multiple references to the same collection. If one reference is used to shrink a collection,
other aliases to that collection must be aware that the length has changed. The extended Index
Checker uses a combination of type qualifiers and method annotations to account for aliasing:
\begin{itemize}
\item
  \textbf{Qualifiers prevent unsafe alias combinations:}
  The type hierarchy ensures that a @GrowOnly reference cannot exist simultaneously with a @Shrinkable
  reference to the \emph{same} collection. This consistency check ensures that no part of the code is
  removing elements from a @GrowOnly List, otherwise, the checker would report a conflict. However,
  an @UnShrinkableRef reference may alias a @Shrinkable one, causing the collection to be shrunk
  through the other alias. The checker handles this scenario by invalidating index information, but
  does not prohibit aliasing.

\item
  \textbf{The @BackedBy annotation:}
  Aids the checker in reasoning about two objects with the same backing storage. This annotation declares
  an aliasing relationship, like the \code{subList} method of \code{List}, indicating that the returned
  list is backed by the original list. In code, it might look like:

\begin{Verbatim}
    public List<E> subList(int from, int to) @BackedBy("this");
\end{Verbatim}

  The Index Checker uses @BackedBy information to propagate mutations in sub-lists, treating
  the original list's length as changed if an element is removed. These explicit alias annotations
  are crucial for soundness and are used in library annotations to inform the checker of
  common aliasing patterns. If your code uses a custom data structure with one collection backed
  by another, @BackedBy can be used similarly.

\item
  \textbf{Method effects on length:}
  The Checker Framework handles index-related facts by identifying methods that can change a
  collection's length. Standard methods like \code{add}, \code{remove}, and \code{clear} are
  length-changing. The framework relies on the qualifier of the receiver or argument, such as
  a @Shrinkable reference, to understand when a method is modifying the length of a structure.
  It also provides a method annotation \code{@ChangesLength} to mark a method as modifying
  the length of a structure. These annotations help the checker assume that the length of the
  specified collection has changed, unless proven otherwise. The JDK's annotated signatures
  use @ChangesLength on any method that can add or remove elements from a collection.

\item
  \textbf{Future pointer analysis:}
  The Index Checker currently doesn't perform whole-program pointer/alias analysis, relying
  on annotations to understand aliasing relationships and mutation effects. Future work
  could integrate pointer analysis to detect when two references might alias the same
  collection or determine which objects a method might modify, reducing the need for manual
  annotations and allowing fine-grained invalidation. Pointer analysis remains an area of
  future work, while the current system balances by requiring programmers to annotate tricky
  aliasing cases.
\end{itemize}

\subsectionAndLabel{Flow-sensitivity and side-effect invalidation}{index-mutable-flow}

Because of mutations, the Index Checker must sometimes forget (invalidate) information that was
established earlier in the code. The checker is \emph{flow-sensitive}: it tracks facts like
“\code{i} is a valid index for \code{list}” within a certain scope, but if something happens
that could falsify that fact, the checker will revoke it.
\begin{itemize}
\item
  \textbf{Invalidation after side effects:}
    The checker takes a conservative approach when a reference is not safe from shrinking
    (i.e., it is not @GrowOnly and not explicitly opted-out), \emph{invalidating all
    index-related qualifiers for that collection}. This means that after a method that might
    mutate the collection's length, any previously inferred type like \code{@IndexFor("list")}
    or \code{@LTLengthOf("list")} is cleared or downgraded. For example, consider this snippet:

      \begin{Verbatim}
        // list is @Shrinkable List<String>
        if (0 <= idx && idx < list.size()) {
          // idx is considered @IndexFor("list") here
          list.remove(0);       // shrink the list
          String s = list.get(idx);  // <-- must recheck, idx may now be out of range
      }
      \end{Verbatim}

    The checker invalidates the valid index \code{idx} after removing a list, potentially reducing its
    length or shifting indices. It warns that \code{list.get(idx)} is unsafe unless the code
    re-establishes the index bound, like checking \code{idx < list.size()}.
\item
  \textbf{The @SideEffectsOnly annotation:}
      The Index Checker uses a method annotation \code{@SideEffectsOnly} to identify methods with
      significant side effects and no relevant return value. This annotation indicates that the
      method's primary purpose is to produce side effects, and any index facts involving objects
      that the method could modify should be invalidated afterward. Unlike @ChangesLength,
      @SideEffectsOnly is a broader signal, as it conservatively assumes that the method may
      have changed the state of any object it had access to. This means that indices previously
      considered safe may need to be re-validated after the call. Library developers can use this
      annotation on void methods or those that mutate structures to ensure soundness. When combined
      with the Purity Checker, this mechanism helps the Index Checker distinguish pure computations
      from mutating ones. A method known to be @Pure (no side effects) will not invalidate index facts
      for its arguments, while a @SideEffectsOnly method will.
\item
  \textbf{Summary of flow rules:}
    Whenever your code performs an operation that might affect a collection’s length:
    \begin{itemize}
    \item
      If the operation is through a @Shrinkable (or unannotated, defaulting to @UnShrinkableRef) reference,
      the checker will drop any knowledge about indices for that collection. You must add new checks
      after the mutation if you want to use indices safely.
    \item
      If the operation is through a @GrowOnly reference, the checker does not need to drop index facts for that collection,
      because the length can only increase (making previously safe indices still safe).
    \item
      If a method call is made and that method is neither known to be pure nor annotated otherwise, the checker errs on the
      side of caution and may invalidate index info for any involved collections. By using @ChangesLength and @SideEffectsOnly
      annotations in the library, this behavior is made more precise, targeting the specific collections that are affected.
    \item
      Code that relies on previously checked indices across mutating calls will trigger warnings, prompting the developer
      to either adjust the code structure or add appropriate re-checks or annotations (such as splitting a single method
      into two phases, or copying the collection size to a local variable before mutation, etc.).
    \end{itemize}
\end{itemize}

\subsectionAndLabel{Using the new annotations in practice}{index-mutable-usage}

The implementation of these features involves adding annotations to both library code (the JDK or third-party collection classes)
and your application code:
\begin{itemize}
\item
  \textbf{Library annotation (JDK):}
    The Java collections library is annotated so that the Index Checker knows how its methods affect lengths. For example,
    \code{List.remove} is annotated to indicate it requires a @Shrinkable receiver (meaning you should have the reference as
    @Shrinkable to call it without error), and it is marked @ChangesLength on the receiver. Methods like \code{add} might be marked
    @ChangesLength as well (to indicate growth), and \code{subList} returns a list annotated with @BackedBy("this"). These annotations
    are trusted by the checker. (The framework does not re-verify the JDK itself; it assumes the library methods behave as annotated.)
    As a result, when you use standard collections, you’ll get appropriate warnings or lack thereof based on these annotations. For
    instance, calling \code{myList.remove(i)} where \code{myList} is not annotated as @Shrinkable will produce an error, because the
    library method signature effectively expects a Shrinkable receiver.

\item
  \textbf{Application code:}
  As an application developer, you primarily use these qualifiers to document and verify your intended mutation patterns.
  Here are some guidelines:
  \begin{itemize}
    \item
      If you never remove elements from a particular collection (you only add or read), you don’t need to write any
      annotation – it will default to @UnShrinkableRef and the checker will assume no removals. If you want to be explicit
      (for documentation), you can annotate it as @UnShrinkableRef or @GrowOnly. Use @GrowOnly if you want to make the stronger
      claim that not only will you not remove, but also that the collection isn’t shrunk via any alias. This could be the
      case for a list that is built up over time but never pruned.
    \item
      If you do remove elements from a collection, annotate the reference (or variable) as \refqualclass{checker/index/qual}{Shrinkable}.
      For example:
        \begin{Verbatim}
          import org.checkerframework.checker.index.qual.Shrinkable;
          @Shrinkable List<String> lines = new ArrayList<>();
          ...
          lines.add("first");
          lines.remove(0);
        \end{Verbatim}
      Marking \code{lines} as @Shrinkable tells the checker that removals are intentional.
      The checker will allow \code{lines.remove(0)} to be called. It will, however, invalidate index facts for
      \code{lines} afterward as discussed. If you omit the @Shrinkable annotation in this scenario, the checker
      would produce an error at the call to \code{remove}, because by default it assumes such calls are not allowed.
    \item
      If you have a reference that you know may be subject to complex mutations that the checker cannot easily reason
      about (perhaps the collection is modified in deeply aliasing structures or via reflection, etc.), you can use
      \refqualclass{checker/index/qual}{UncheckedShrinkable} as a last resort. For example,
      \code{@UncheckedShrinkable List<Foo> legacyList;}. This will suppress index-out-of-bounds errors for operations
      on \code{legacyList}, under the assumption that you will manage its safety manually or via tests. Use this
      sparingly – it’s better to design your code such that @Shrinkable and @UnShrinkableRef/@GrowOnly references are sufficient.
    \item
      When writing new APIs or methods in your code, consider whether they mutate collections. If a method’s sole
      purpose is to mutate a collection, you might mark it as @SideEffectsOnly and/or @ChangesLength on the
      relevant parameters. For example:
        \begin{Verbatim}
          @SideEffectsOnly
          void trimList(@Shrinkable List\<?> list) { ... remove elements from list ... }
        \end{Verbatim}
      Annotating \code{list} as @Shrinkable makes it clear the method might shrink it, and @SideEffectsOnly
      on the method means callers should not expect any index relationships to remain valid after calling
      \code{trimList}. The Index Checker will enforce these expectations at call sites.
    \item
      The @BackedBy annotation would mostly appear in library or framework code. In your own code, you might use
      it if you implement a view onto a collection. For instance, if you wrote a custom class \code{FilteredList}
      that presents a filtered view of an underlying list, you could annotate \code{FilteredList} with @BackedBy
      to link it to the original. This will help the checker understand that a mutation to the filtered view
      affects the base list.
  \end{itemize}
\item
  \textbf{Example:}
    Suppose you maintain two lists that should always have the same length (parallel arrays pattern).
    You can use the Index Checker’s existing same-length mechanism (the @SameLen annotation)
    \emph{together} with the new qualifiers. You might declare:

    \begin{Verbatim}
      List<String> @SameLen("ids") @Shrinkable names;
          List<Integer> @Shrinkable ids;
        \end{Verbatim}
      and ensure that you perform removals on both lists in tandem. The @SameLen("ids") on \code{names}
      tells the Index Checker that \code{names} and \code{ids} start with the same length, but because
      they are Shrinkable, you must also ensure any removal keeps them in sync. The checker will
      invalidate index info on both if one is mutated, so you would re-establish the relationship
      after any modifications (perhaps by asserting their sizes match within the code). In future, the
      checker might provide a more automated way to handle such “linked-list” mutations (see below).

    \item
      \textbf{Implementation notes:}
        Under the hood, the Index Checker enforces these annotations by extending the type rules.
        For instance, the method invocation rule checks the receiver’s qualifier against what the
        method requires: calling a method that is known to shrink the list will produce an error if
        the receiver is not @Shrinkable (or @UncheckedShrinkable). Additionally, the dataflow
        (flow-sensitive analysis) component of the checker is augmented to drop information when it
        encounters a method call or mutation on a potentially-shrinking reference. These changes are
        largely transparent to users – you experience them as error messages or lack thereof.
    \end{itemize}
    Overall, using the new system typically involves adding a few well-placed annotations. The
    most common use case will be marking a variable as @Shrinkable in sections of code that
    remove elements from a collection. In exchange, you gain confidence (via compiler errors or
    lack of them) that you have correctly handled index-safety even as your collections change size.

    \subsectionAndLabel{Advanced features and future improvements}{index-mutable-future}

    \begin{itemize}
    \item
      \textbf{Leveraging uniqueness for safe mutation:}
        A \emph{unique} reference can make shrinking a collection safer by allowing the programmer
        to immediately adjust or re-check indices after removal. This feature can be automatic
        between mutation modes, allowing for full mutation of a newly created list. The checker,
        in conjunction with an Aliasing Checker, can determine if a list is unique, treat it as
        @Shrinkable during the building phase, and then convert or restrict it to @UnShrinkableRef
        once published. This allows for free mutation when safe, with sound checking for aliases.

    \item
      \textbf{Alternate qualifier hierarchies:}
        The subtyping hierarchy for GrowOnly and Shrinkable references has been chosen, but other
        designs were considered. These could be two parallel dimensions with a common supertype or
        @GrowOnly as a supertype of @Shrinkable. The chosen hierarchy enforces constraints without
        burdening the developer, ensuring no @GrowOnly can be mistakenly used in place of @Shrinkable.
        The developer manual discusses some alternate designs and why they were not adopted.
        The top design, @UnShrinkableRef, introduces @UncheckedShrinkable as a subtype of @Shrinkable.

    \item
      \textbf{Same-length collection invariants:}
        The Index Checker already supports a @SameLen annotation for fixed-size sequences, indicating
        equal length between two sequences. However, it's a topic of interest to extend this concept
        to mutable sequences, such as two lists that are always mutated together. A future enhancement
        could introduce a mechanism to declare this relationship more directly for shrinkable lists,
        such as an effect annotation or a stronger guarantee that a mutation on one triggers an automatic
        mutation on the other. This would require more sophisticated analysis or integration with a
        relational invariant checker.

    \item
      \textbf{Unified effect and purity annotations:}
        The introduction of @ChangesLength and @SideEffectsOnly raises the question of integrating these
        with existing effect systems like the Purity Checker's @Pure and the generalized effect system
        for determinism or thread-safety. A unified effect system could reduce duplication of annotations
        and provide a single framework to reason about side effects, with the Index Checker being one
        consumer of that information. Balancing precision with usability is a complex undertaking, but it
        could make the framework more powerful and general. Future developments could include a broader
        category of "effects on data structures" or integration with a more universal effect tracking mechanism.

    \item
      \textbf{Simpler heuristic checks for side effects:}
        The current approach to analyzing index mutability in code is conservative due to the lack of
        full pointer analysis. Future work could introduce simpler, targeted analyses to catch common
        misuse patterns without a full alias analysis. For example, if a previously-validated index
        is used after a method call that takes a collection as a parameter, warning the developer that
        the index might no longer be valid could heuristically catch likely bugs even when annotations
        are missing or incomplete. Integrating with the Must Call/Must Release framework could ensure
        post-conditions for shrunk sequences. These "simple effect checks" could provide guidance in
        code that hasn't been fully annotated for index mutability.
    \end{itemize}


\sectionAndLabel{Technical papers}{index-papers}

The paper ``Lightweight Verification of Array Indexing'' (ISSTA 2018,
\myurl{https://homes.cs.washington.edu/~mernst/pubs/array-indexing-issta2018-abstract.html})
gives more details about the Index Checker.
``Enforcing correct array indexes with a type system''~\cite{Santino2016} (FSE 2016) describes
an earlier version.


%%  LocalWords:  NegativeArraySizeException pre myArray IndexFor someArray
%%  LocalWords:  MyJavaFile LTLengthOf LTEqLengthOf GTENegativeOne GTE str
%%  LocalWords:  LowerBoundUnknown LTOMLengthOf LTEq LTOM UpperBoundBottom
%%  LocalWords:  UpperBoundUnknown MinLen MinLenBottom SameLen indexOf abc
%%  LocalWords:  SameLenUnknown SameLenBottom lastIndexOf html lang charAt
%%  LocalWords:  LengthOf IndexOrLow PolyIndex PolyLowerBound PolyLength
%%  LocalWords:  PolyUpperBound PolySameLen PolyValue LowerBoundBottom
%%  LocalWords:  lowerbound upperbound EnsuresLTLengthOf targetValue
%%  LocalWords:  EnsuresLTLengthOfIf boolean HasSubsequence LessThan
%%  LocalWords:  minlen ArrayLenRange ArrayLen EnsuresMinLenIf samelen
%%  LocalWords:  searchindex binarySearch SearchIndexFor NegativeIndexFor
%%  LocalWords:  SearchIndex bitwise SearchIndexBottom SearchIndexUnknown
%%  LocalWords:  Substring substringindex substring SubstringIndexFor
%%  LocalWords:  SubstringIndexBottom SubstringIndexUnknown LessThanBottom
%%  LocalWords:  LessThanUnknown typecheck typechecks system''
% LocalWords:  UpperBoundLiteral
